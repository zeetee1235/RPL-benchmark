\documentclass[a4paper,11pt]{article}
\usepackage[utf8]{inputenc}
\usepackage[korean]{babel}
\usepackage{kotex}
\usepackage{graphicx}
\usepackage{booktabs}
\usepackage{float}
\usepackage{subcaption}
\usepackage{hyperref}
\usepackage[margin=2.5cm]{geometry}

\title{RPL 프로토콜 성능 평가 실험 결과}
\author{WSN-IoT Lab}
\date{\today}

\begin{document}

\maketitle

\section{실험 개요}

본 실험은 Contiki-NG 환경에서 RPL(Routing Protocol for Low-Power and Lossy Networks) 프로토콜의 
성능을 평가하기 위해 수행되었다. 세 가지 RPL 변형 프로토콜을 비교 분석하였다:

\begin{itemize}
    \item \textbf{RPL-Classic}: 표준 RPL 구현 (Storing Mode)
    \item \textbf{RPL-Lite}: 경량화된 RPL 변형 (Non-Storing Mode)
    \item \textbf{BRPL}: 버퍼 인식 RPL (Buffer-Aware RPL)
\end{itemize}

\section{실험 환경}

\subsection{네트워크 구성}
\begin{itemize}
    \item 시뮬레이터: Cooja (Contiki-NG)
    \item 노드 수: 5, 10, 15, 20개 센서 노드 + 1개 루트 노드
    \item 전송 간격: 10초
    \item 시뮬레이션 시간: 600초 (10분)
    \item 반복 횟수: 각 구성당 3회 (seed 1-3)
\end{itemize}

\subsection{성능 지표}
\begin{itemize}
    \item \textbf{PDR (Packet Delivery Ratio)}: 전송된 패킷 중 수신된 패킷의 비율
    \item \textbf{Delay}: 패킷 전송 지연 시간 (밀리초)
\end{itemize}

\section{Stage 1 실험 결과}

Stage 1은 기본 네트워크 환경에서의 성능 평가로, 간섭 없이 정상적인 조건에서 
각 프로토콜의 성능을 측정하였다.

\subsection{통계 요약}

\begin{table}[H]
\centering
\caption{Performance Comparison of RPL Variants (Stage 1)}
\label{tab:stage1_results}
\begin{tabular}{llrrr}
\toprule
Mode & N Senders & PDR (\%) & Delay (ms) & Runs \\
\midrule
rpl-classic & 5 & 100.00 $\\pm$ 0.00 & 368.17 $\\pm$ 55.95 & 3 \\
rpl-classic & 10 & 100.00 $\\pm$ 0.00 & 379.95 $\\pm$ 46.12 & 3 \\
rpl-classic & 15 & 100.00 $\\pm$ 0.00 & 374.22 $\\pm$ 11.26 & 3 \\
rpl-classic & 20 & 100.00 $\\pm$ 0.00 & 374.74 $\\pm$ 21.29 & 3 \\
rpl-classic & 25 & 100.00 $\\pm$ 0.00 & 389.58 $\\pm$ 22.22 & 3 \\
rpl-classic & 30 & 100.00 $\\pm$ 0.00 & 380.34 $\\pm$ 19.35 & 3 \\
rpl-classic & 40 & 100.00 $\\pm$ 0.00 & 370.57 $\\pm$ 2.98 & 3 \\
rpl-classic & 50 & 100.00 $\\pm$ 0.00 & 419.01 $\\pm$ 25.20 & 3 \\
brpl & 5 & 0.00 $\\pm$ 0.00 & 0.00 $\\pm$ 0.00 & 3 \\
brpl & 10 & 0.00 $\\pm$ 0.00 & 0.00 $\\pm$ 0.00 & 3 \\
brpl & 15 & 0.00 $\\pm$ 0.00 & 0.00 $\\pm$ 0.00 & 3 \\
brpl & 20 & 0.00 $\\pm$ 0.00 & 0.00 $\\pm$ 0.00 & 3 \\
brpl & 25 & 0.00 $\\pm$ 0.00 & 0.00 $\\pm$ 0.00 & 3 \\
brpl & 30 & 0.00 $\\pm$ 0.00 & 0.00 $\\pm$ 0.00 & 3 \\
brpl & 40 & 0.00 $\\pm$ 0.00 & 0.00 $\\pm$ 0.00 & 3 \\
brpl & 50 & 0.00 $\\pm$ 0.00 & 0.00 $\\pm$ 0.00 & 3 \\
\bottomrule
\end{tabular}
\end{table}


\subsection{성능 그래프}

\begin{figure}[H]
    \centering
    \includegraphics[width=0.8\textwidth]{figures/stage1_pdr.pdf}
    \caption{노드 수에 따른 Packet Delivery Ratio (PDR) 비교}
    \label{fig:stage1_pdr}
\end{figure}

\begin{figure}[H]
    \centering
    \includegraphics[width=0.8\textwidth]{figures/stage1_delay.pdf}
    \caption{노드 수에 따른 평균 지연 시간 비교}
    \label{fig:stage1_delay}
\end{figure}

\subsection{결과 분석}

\subsubsection{Packet Delivery Ratio (PDR)}
그림 \ref{fig:stage1_pdr}에서 보듯이, 노드 수가 증가함에 따라 세 프로토콜 모두 
PDR이 변화하는 것을 확인할 수 있다. 

\begin{itemize}
    \item \textbf{RPL-Classic}: Storing 모드로 동작하여 각 노드가 라우팅 테이블을 유지
    \item \textbf{RPL-Lite}: Non-Storing 모드로 경량화되어 메모리 효율적
    \item \textbf{BRPL}: 큐 상태를 고려한 라우팅으로 혼잡 상황에서 유리
\end{itemize}

\subsubsection{평균 지연 시간}
그림 \ref{fig:stage1_delay}는 노드 수에 따른 평균 패킷 지연 시간을 보여준다.
네트워크 부하가 증가할수록 지연 시간도 증가하는 경향을 보인다.

\section{결론}

본 실험을 통해 세 가지 RPL 변형 프로토콜의 성능을 체계적으로 비교 분석하였다.
각 프로토콜은 서로 다른 장단점을 가지고 있으며, 네트워크 환경과 요구사항에 따라
적합한 프로토콜을 선택할 수 있다.

\subsection{주요 발견사항}
\begin{itemize}
    \item 노드 수가 증가할수록 네트워크 부하가 증가하여 성능 저하 발생
    \item 각 프로토콜은 서로 다른 trade-off 관계를 보임
    \item BRPL은 버퍼 인식 라우팅으로 혼잡 상황에서 이점
\end{itemize}

\subsection{향후 연구}
\begin{itemize}
    \item Stage 2: 간섭 환경에서의 성능 평가
    \item Stage 3: 동적 네트워크 환경에서의 안정성 평가
    \item 에너지 소비량 분석
    \item 실제 하드웨어 테스트베드 검증
\end{itemize}

\end{document}
