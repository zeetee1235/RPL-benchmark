\documentclass[11pt]{article}
\usepackage[utf8]{inputenc}
\usepackage[T1]{fontenc}
\usepackage[margin=1in]{geometry}
\usepackage{graphicx}
\usepackage{booktabs}
\usepackage{float}
\usepackage{hyperref}
\usepackage{kotex}
\usepackage{lmodern}

\title{RPL-lite와 BRPL 성능 비교 실험 보고서\\
(Headless Cooja 시뮬레이션)}
\date{2026년 1월 16일}

\begin{document}
\maketitle

\section*{주요 실험 결과}
\begin{itemize}
  \item Headless Cooja 시뮬레이션 파이프라인이 안정적으로 동작하며, RTT 로그가 일관되게 수집되었습니다.
  \item 본 실험에서 rpl-lite는 2단계 및 3단계에서 성능 저하(collapse)를 보인 반면, BRPL은 아직 성능 저하가 관찰되지 않았습니다.
  \item 현재 설정된 스트레스 조건이 충분히 강하지 않아 최종 결론을 도출하기에는 제한적이며, BRPL에 대한 추가 검증이 필요합니다.
  \item 더 높은 수준의 스트레스 조건이 필요하나, 실행 시간이 병목 현상을 보이고 있어 최적화가 필요합니다.
  \item BRPL이 더 안정적일 가능성이 있으나, 현재 조건이 충분히 가혹하지 않거나 OF(Objective Function)의 혼잡도 계산이 제한적일 수 있습니다.
\end{itemize}

\section*{요약 표}
\begin{table}[H]
\centering
\caption{임계점 요약}
\label{tab:thresholds}
\begin{tabular}{lllp{6cm}rr}
\toprule
모드 & 단계 & 발견 & 조건 & PDR (중앙값) & RTT (중앙값 ms) \\
\midrule
brpl & stage1 & N & - & - & - \\
brpl & stage2 & N & - & - & - \\
brpl & stage3 & N & - & - & - \\
rpl-lite & stage1 & N & - & - & - \\
rpl-lite & stage2 & Y & N=50, sr=1, ir=0.95, si=10 & 0.888 & 599.0 \\
rpl-lite & stage3 & Y & N=25, sr=0.85, ir=0.9, si=2 & 0.879 & 591.0 \\
\bottomrule
\end{tabular}
\end{table}

\begin{table}[H]
\centering
\caption{단계별 평균 성능}
\label{tab:stage_means}
\begin{tabular}{llrr}
\toprule
모드 & 단계 & 평균 PDR & 평균 RTT (ms) \\
\midrule
brpl & stage1 & 0.995 & 121.6 \\
rpl-lite & stage1 & 0.957 & 113.2 \\
brpl & stage2 & 0.972 & 211.3 \\
rpl-lite & stage2 & 0.890 & 212.2 \\
brpl & stage3 & 0.993 & 195.9 \\
rpl-lite & stage3 & 0.909 & 235.8 \\
\bottomrule
\end{tabular}
\end{table}


\section*{붕괴 시점 탐지 기준}
\begin{itemize}
  \item 단일 run 붕괴: \texttt{PDR < 0.90} 또는 \texttt{avg\_delay\_ms > 5000}(RTT 기반), 또는 \texttt{invalid\_run=1}.
  \item \texttt{p95\_rtt\_ms}가 존재하면 \texttt{p95\_rtt\_ms > 8000} 기준을 우선 적용.
  \item 조건별 붕괴: seed 다수결 기준(\texttt{collapse\_frac} $\ge$ 2/3).
  \item 스테이지 정렬 규칙에 따라 최초로 붕괴가 성립한 조건을 붕괴 시점으로 기록.
\end{itemize}

\section*{1단계 실험 결과}
\paragraph{실험 설계}
송신 노드 수(N)를 증가시키며 링크 품질은 고정(성공/간섭 비율 1.0, 간격 10s)한다.
각 N은 seed 1--3으로 반복 수행하며, rpl-lite와 brpl을 동일 조건에서 비교한다.
\begin{figure}[H]
  \centering
  \includegraphics[width=0.7\textwidth]{figures/stage1_pdr.pdf}
  \caption{1단계: 송신 노드 수에 따른 PDR 변화 (rpl-lite 대 brpl 비교)}
\end{figure}

\section*{2단계 및 3단계 실험 결과}
\paragraph{2단계 설계}
Stage 1 rpl-lite 결과에서 안정(Stable)과 경계(Marginal) N을 자동 선택한 뒤,
링크 품질 파라미터(success/interference ratio)를 스윕하여 붕괴 조건을 탐색한다.
\paragraph{3단계 설계}
Stage 2 rpl-lite 결과에서 knee 조건을 선택하고, 트래픽 부하를 높이기 위해
send\_interval\_s를 단계적으로 감소시킨다.
\begin{figure}[H]
  \centering
  \begin{minipage}{0.48\textwidth}
    \centering
    \includegraphics[width=\textwidth]{figures/stage2_rtt_at_collapse.pdf}
    \caption{2단계: 임계 조건에서의 RTT 비교}
  \end{minipage}
  \hfill
  \begin{minipage}{0.48\textwidth}
    \centering
    \includegraphics[width=\textwidth]{figures/stage3_rtt_at_collapse.pdf}
    \caption{3단계: 임계 조건에서의 RTT 비교}
  \end{minipage}
\end{figure}

\section*{제어 오버헤드와 성능의 관계}
\begin{figure}[H]
  \centering
  \includegraphics[width=0.8\textwidth]{figures/overhead_vs_performance.pdf}
  \caption{제어 오버헤드(DIO+DAO) 대 PDR 및 RTT의 관계}
\end{figure}

\section*{파라미터 커버리지 분석}
\begin{figure}[H]
  \centering
  \includegraphics[width=0.8\textwidth]{figures/stage2_coverage_heatmap.pdf}
  \caption{2단계 파라미터 커버리지 히트맵}
\end{figure}

\section*{한계 및 향후 계획}
\begin{itemize}
  \item BRPL은 본 스윕 범위 내에서 붕괴가 관측되지 않았으므로, 더 가혹한 조건에서의 추가 검증이 필요하다.
  \item 스트레스 범위를 확장하려면 실행 시간이 병목이 되므로, 병렬화/캐시 활용 등 실행 최적화가 필요하다.
  \item BRPL의 혼잡 반영이 제한적일 가능성이 있어, OF 파라미터 스윕과 로그 기반 검증을 추가할 계획이다.
\end{itemize}

\end{document}
